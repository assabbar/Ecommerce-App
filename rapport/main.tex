

\documentclass[12pt,a4paper]{report}


% Page de garde adaptée au projet E-Commerce DevOps
\usepackage[utf8]{inputenc}
\begin{document}

\begin{titlepage}
\begin{center}
    \vspace*{1cm}
    \begin{minipage}{1\textwidth}
        \usepackage[utf8]{inputenc}
        \usepackage[T1]{fontenc}
        \usepackage{lmodern}
        \usepackage{graphicx}
        \usepackage{hyperref}
        \usepackage{listings}
        \usepackage{xcolor}
        \usepackage{geometry}
        \usepackage{float}
        \usepackage{amsmath}
        \usepackage{amssymb}
        \usepackage{enumitem}
        \usepackage{newunicodechar}
        \centering
        % Gestion des caractères Unicode courants
        \newunicodechar{↔}{\ensuremath{\leftrightarrow}}
        \newunicodechar{#}{\#}
        \vspace{1cm}
        % Marges
        \geometry{
         a4paper,
         total={170mm,257mm},
         left=20mm,
         top=20mm,
        }
        {\Huge \textbf{Plateforme E-Commerce DevOps}}
        % Style listings
        \definecolor{codegreen}{rgb}{0,0.6,0}
        \definecolor{codegray}{rgb}{0.5,0.5,0.5}
        \definecolor{codepurple}{rgb}{0.58,0,0.82}
        \definecolor{backcolour}{rgb}{0.95,0.95,0.92}
        \vspace{1cm}
        \lstdefinestyle{mystyle}{
            backgroundcolor=\color{backcolour},   
            commentstyle=\color{codegreen},
            keywordstyle=\color{magenta},
            numberstyle=\tiny\color{codegray},
            stringstyle=\color{codepurple},
            basicstyle=\ttfamily\footnotesize,
            breakatwhitespace=false,         
            breaklines=true,                 
            captionpos=b,                    
            keepspaces=true,                 
            numbers=left,                    
            numbersep=5pt,                  
            showspaces=false,                
            showstringspaces=false,
            showtabs=false,                  
            tabsize=2
        }
        \lstset{style=mystyle}
        
        {\Large \textbf{Rapport Technique Complet}}
        
        % Logo ENSAH (adapter le chemin si besoin)
        \vspace{1.5cm}
        \includegraphics[width=0.18\textwidth]{imgs/ENSAH.png}
        \vspace{2cm}
        
        {\large Réalisé par : \textbf{Malak ASSABBAR}}
        
        {\large Encadré par : Pr. Bahiri}
        
        \vspace{1cm}
        {\large École Nationale des Sciences Appliquées d'Al Hoceima (ENSAH)}
        \\
        {\large Année universitaire : 2025/2026}
    \end{minipage}
\end{center}
\end{titlepage}

\newpage

    ableofcontents
\newpage




% Police uniforme : lmodern
\chapter{Présentation du projet}

\section{Introduction}
Le projet présente une plateforme e-commerce basée sur une architecture microservices moderne destinée à fournir une expérience d'achat scalable, résiliente et maintenable. Le frontend est développé en \textbf{Angular 18} tandis que le backend est composé de services Java utilisant \textbf{Spring Boot 3} (Java 21). L'ensemble est containerisé avec \textbf{Docker}, orchestré avec \textbf{Kubernetes} et automatisé via \textbf{Terraform} et des pipelines CI/CD. \textbf{L'application est hébergée sur Microsoft Azure, déployée dans un cluster Kubernetes (AKS) pour garantir la haute disponibilité, la sécurité et la scalabilité.}

\section{Objectifs}
- Fournir une architecture modulaire permettant l'évolution indépendante des services.
- Assurer la haute disponibilité et la résilience via l'orchestration Kubernetes.
- Mettre en place une chaîne CI/CD reproductible et sécurisée pour accélérer les livraisons.
- Garantir l'observabilité complète (métriques, logs, traces) et la gestion centralisée des secrets.

\section{Vue d'ensemble de l'architecture}
Le flux de données général est le suivant :
\begin{enumerate}
  \item L'utilisateur navigue depuis l'interface Angular.
  \item Les requêtes sont routées via l'API Gateway qui gère l'authentification et le routage vers les microservices.
  \item Les microservices (product, order, inventory, notification) traitent la logique métier et communiquent éventuellement via des messages asynchrones.
  \item La supervision collecte métriques et logs (Prometheus, Grafana, Loki) et déclenche des alertes via AlertManager.
  \item L'infrastructure est provisionnée et versionnée via Terraform.
\end{enumerate}

\section{Cas d'utilisation}
- Utilisateur : navigation catalogue, ajout au panier, validation de commande, suivi.
- Administrateur : gestion des produits et stocks, surveillance des indicateurs et gestion des incidents.

\section{Technologies principales}

\subsection{Frontend}
\begin{table}[H]
\centering
\begin{tabular}{|l|p{9cm}|}
\hline
	extbf{Technologie} & \textbf{Rôle dans le projet} \\
\hline
Angular 18 & Framework principal pour le développement de l'interface utilisateur SPA, gestion du routage, des composants et de l'état. \\
TypeScript & Langage typé pour la robustesse, la maintenabilité et la sécurité du code frontend. \\
Tailwind CSS & Framework utilitaire pour le design responsive et la personnalisation rapide de l'UI. \\
Karma/Jasmine & Outils de tests unitaires pour garantir la qualité du code frontend. \\
Cypress & Tests end-to-end automatisés pour valider les parcours critiques utilisateur. \\
\hline
\end{tabular}
\end{table}

\subsection{Backend}
\begin{table}[H]
\centering
\begin{tabular}{|l|p{9cm}|}
\hline
	extbf{Technologie} & \textbf{Rôle dans le projet} \\
\hline
Java 21 & Langage principal pour la logique métier, la performance et la sécurité. \\
Spring Boot 3 & Framework pour la création de microservices REST, gestion de la configuration, des dépendances et de la sécurité. \\
Spring Data JPA & Accès et mapping aux bases de données relationnelles (MySQL). \\
Spring Data MongoDB & Accès aux données NoSQL pour certains services. \\
Spring Security & Authentification et autorisation (JWT, rôles). \\
Maven & Build, gestion des dépendances, packaging et exécution des tests. \\
JUnit/Mockito & Tests unitaires et d'intégration backend. \\
Testcontainers & Tests d'intégration avec bases de données et services réels en conteneur. \\
\hline
\end{tabular}
\end{table}

\subsection{Infrastructure}
\begin{table}[H]
\centering
\begin{tabular}{|l|p{9cm}|}
\hline
	extbf{Technologie} & \textbf{Rôle dans le projet} \\
\hline
Docker & Containerisation des applications pour portabilité et cohérence des environnements. \\
Kubernetes (AKS) & Orchestration, déploiement, scalabilité et gestion des microservices sur Azure. \\
Helm & Gestion des packages Kubernetes, déploiement automatisé et versionné. \\
Terraform & Infrastructure as Code pour provisionner et gérer toutes les ressources Azure. \\
Azure Container Registry (ACR) & Stockage et versionnement des images Docker. \\
Azure Key Vault & Gestion centralisée et sécurisée des secrets et credentials. \\
Azure Storage & Stockage des fichiers statiques et images produits. \\
MySQL Flexible Server & Base de données transactionnelle pour les commandes et l'inventaire. \\
\hline
\end{tabular}
\end{table}

\subsection{Monitoring et Observabilité}
\begin{table}[H]
\centering
\begin{tabular}{|l|p{9cm}|}
\hline
	extbf{Technologie} & \textbf{Rôle dans le projet} \\
\hline
Prometheus & Collecte des métriques applicatives et système (scraping des endpoints Actuator). \\
Grafana & Visualisation des métriques, création de dashboards et alerting. \\
Loki & Centralisation et recherche des logs applicatifs structurés. \\
Tempo & Tracing distribué pour l’analyse des performances et la corrélation des requêtes. \\
\hline
\end{tabular}
\end{table}

\subsection{CI/CD}
\begin{table}[H]
\centering
\begin{tabular}{|l|p{9cm}|}
\hline
	extbf{Technologie} & \textbf{Rôle dans le projet} \\
\hline
Jenkins & Orchestration des pipelines CI/CD, automatisation des builds, tests et déploiements. \\
Docker Registry (ACR) & Stockage sécurisé des images Docker pour déploiement sur AKS. \\
Helm charts & Déploiement automatisé et versionné des applications sur Kubernetes. \\
SonarQube & Analyse de la qualité et de la sécurité du code. \\
Trivy/Snyk & Scans de vulnérabilités sur les images et dépendances. \\
\hline
\end{tabular}
\end{table}



% Police uniforme : lmodern
\chapter{Backend}

\section{Stack Technique}


\begin{itemize}
	\item \textbf{Java 21}~: Langage principal du backend, apporte performance, sécurité et compatibilité avec les frameworks modernes.
	\item \textbf{Spring Boot 3.2+}~: Framework principal pour créer des microservices REST, facilite la configuration, l’injection de dépendances et l’exposition d’APIs robustes.
	\item \textbf{Spring Cloud}~: Fournit des outils pour la configuration centralisée, la découverte de services (Eureka), la gestion de la résilience (circuit breaker) et la communication inter-services.
	\item \textbf{Spring Data JPA/MongoDB}~: Simplifie l’accès aux bases de données relationnelles (MySQL) et NoSQL (MongoDB), gère le mapping objet-relationnel et les requêtes complexes.
	\item \textbf{Spring Security}~: Assure l’authentification (JWT) et l’autorisation (rôles, droits d’accès) sur toutes les APIs.
	\item \textbf{Spring Validation}~: Permet de valider les entrées utilisateurs et les objets métiers pour garantir l’intégrité des données.
	\item \textbf{Maven 3.9+}~: Outil de build et de gestion des dépendances, utilisé pour compiler, tester et packager les microservices.
	\item \textbf{Kubernetes}~: Orchestrateur de conteneurs, déploie et gère automatiquement les microservices sur le cloud Azure (AKS).
	\item \textbf{Docker}~: Permet la containerisation des applications pour garantir la portabilité et la cohérence entre les environnements.
	\item \textbf{SLF4J + Logback}~: Bibliothèques de logging pour la journalisation structurée et la traçabilité des événements applicatifs.
	\item \textbf{Jackson}~: Bibliothèque de sérialisation/désérialisation JSON, utilisée pour les échanges de données entre services et avec le frontend.
\end{itemize}

\section{Configuration Spring}

\subsection{Application.yml}
Profils : dev, staging, prod. Database configuration : URL, credentials (from Key Vault via env vars), pool size (HikariCP), timeout, retry policy. Server port 8080. Logging level INFO (DEBUG dev). JWT secret (from Key Vault). Actuator endpoints /actuator/health/live, /actuator/health/readiness.

\subsection{Spring Security}

Spring Security assure la protection des APIs et la gestion des accès dans l’ensemble du backend. Les principaux mécanismes mis en place sont :
\begin{itemize}
	\item \textbf{Filtre d’authentification JWT personnalisé} : chaque requête entrante passe par un filtre qui vérifie la présence et la validité d’un token JWT. Cela permet de garantir que seuls les utilisateurs authentifiés peuvent accéder aux ressources protégées.
	\item \textbf{Autorisation basée sur les rôles (ADMIN, USER)} : les endpoints sont sécurisés selon le rôle de l’utilisateur. Par exemple, certaines opérations (création, suppression) sont réservées aux administrateurs.
	\item \textbf{Configuration CORS restreinte à l’API Gateway} : seules les requêtes provenant du domaine de l’API Gateway sont acceptées, ce qui limite les risques d’attaques cross-origin.
	\item \textbf{CSRF désactivé (stateless JWT)} : comme l’authentification repose sur des tokens JWT et non sur des sessions, la protection CSRF est désactivée pour simplifier la gestion des requêtes et éviter les faux positifs.
	\item \textbf{Encodage des mots de passe avec bcrypt (force 12)} : tous les mots de passe sont stockés de façon sécurisée grâce à l’algorithme bcrypt avec un facteur de complexité élevé, ce qui protège contre les attaques par force brute.
	\item \textbf{Contexte de sécurité par requête (thread-local)} : chaque requête dispose de son propre contexte de sécurité, isolé des autres, pour garantir la confidentialité et l’intégrité des données utilisateur.
\end{itemize}

\subsection{JPA Configuration}
Hibernate dialect MySQL8Dialect (CosmosDB Mongo dialect optionnel). DDL mode validate (schema pre-created). Lazy loading LAZY collections. Batch size 20 inserts/updates. Statistics optionnel (dev logging).

\section{API Gateway Service}
\subsection{Présentation générale}
Le service API Gateway constitue le point d'entrée unique pour toutes les requêtes clients. Il a pour rôle principal la gestion de l'authentification, du routage, de la sécurisation des endpoints et de la mise en place de politiques transverses (rate limiting, logging, traçage). Le Gateway implémente des filtres et des policies pour enrichir les requêtes (injection de headers, vérification JWT) et centraliser la gestion des erreurs et des retraits (circuit breaker).

\subsection{Fonctionnalités principales}
\begin{itemize}
	\item Routage dynamique vers les microservices (products, orders, inventory, notifications).
	\item Authentification et validation des tokens JWT.
	\item Application de politiques de sécurité (CORS, rate limiting, IP allowlist).
	\item Collecte des métriques et propagation des traces pour l'observabilité.
	\item Fallbacks et réponses standardisées en cas d'erreur upstream.
\end{itemize}

\subsection{Endpoints}
GET /api/health (public, no auth). GET /api/routes (public, list services). POST /api/auth/login (email, password). POST /api/auth/register (email, password, name). POST /api/auth/refresh (JWT refresh token).

\subsection{Routing Rules}
/api/products/* → ProductService:8081. /api/orders/* → OrderService:8082. /api/inventory/* → InventoryService:8083. /api/notifications/* → NotificationService:8084. Load balancing round-robin Kubernetes Service. Timeout 30s per request. Retry logic 3 attempts transient errors.

\subsection{Rate Limiting}
Per user per IP : 100 requests/min. Per endpoint : /api/auth/login 5 attempts/min (brute force protection).429 Too Many Requests response.

\subsection{Circuit Breaker}
Hystrix/Resilience4j : trip circuit si 50\% requests fail 5 successive calls. Timeout default 3s per service call. Fallback response optionnel.

\section{Product Service}
\subsection{Présentation générale}
Le Product Service gère le catalogue produit et toutes les données associées (descriptions, prix, images, catégories). Il est optimisé pour des opérations de lecture fréquentes et des écritures modérées. Le service expose des APIs REST pour la consultation et la gestion des produits, et publie des événements (ProductCreated, ProductUpdated) pour notifier les autres services (indexation, cache, inventaire).

\subsection{Fonctionnalités principales}
\begin{itemize}
	\item CRUD complet sur les produits et catégories.
	\item Recherche filtrée et pagination, avec support des facettes.
	\item Upload et gestion d'images via le Storage (Azure Blob).
	\item Publication d'événements métier lors de changements produits.
	\item Caching côté lecture pour réduire la latence (Redis optionnel).
\end{itemize}

\subsection{Endpoints}
GET /api/products (paginated, filters). GET /api/products/:id. POST /api/products (admin only). PUT /api/products/:id (admin). DELETE /api/products/:id (admin). GET /api/products/search (keyword, filters). GET /api/categories. POST /api/categories (admin).

\subsection{Entités JPA}
Product : id (PK), name, description, price, discount\%, category\_id (FK), sku, created\_at, updated\_at, status (ACTIVE/INACTIVE). Category : id, name, description.

\subsection{Repositories}
ProductRepository extends JpaRepository\<Product, Long\>. Custom methods : findByNameContaining, findByCategoryId, findByPriceBetween. Query annotations optionnel pour complex queries.

\subsection{Services}
ProductService : business logic CRUD, search, filtering. Validations : name not blank, price > 0, sku unique. Cache : @Cacheable on getProductById (invalidate @CacheEvict on update). Event publish ProductCreated/Updated.

\section{Order Service}
\subsection{Présentation générale}
Le Order Service orchestre la création et le suivi des commandes. Il coordonne la vérification du panier, le calcul des totaux, la validation du paiement et l'émission d'événements asynchrones pour déclencher la réservation de stock et l'envoi de notifications. Le service est conçu pour garantir la consistance via des patterns de saga/compensation.

\subsection{Fonctionnalités principales}
\begin{itemize}
	\item Création de commande avec validation business (prix, disponibilité).
	\item Gestion du cycle de vie d'une commande (PENDING, CONFIRMED, SHIPPED, DELIVERED, CANCELLED).
	\item Intégration paiement (adapter pattern pour gateways externes).
	\item Publication d'événements (OrderCreated, PaymentFailed) pour orchestration asynchrone.
	\item API pour consultation historique et tracking utilisateur.
\end{itemize}

\subsection{Endpoints}
POST /api/orders (create order, authenticated). GET /api/orders (user orders). GET /api/orders/:id (detail). PUT /api/orders/:id/status (admin). DELETE /api/orders/:id (cancel, if PENDING). GET /api/orders/:id/tracking.

\subsection{Entités}
Order : id, user\_id, order\_date, total\_amount, tax\_amount, status (PENDING/CONFIRMED/SHIPPED/DELIVERED/CANCELLED), billing\_address, shipping\_address, created\_at, updated\_at. OrderItem : id, order\_id (FK), product\_id (FK), quantity, unit\_price, subtotal. Address : street, city, zipcode, country, user\_id (FK).

\subsection{Saga Pattern}
Order creation saga : CreateOrderCommand → OrderService (validate cart, calculate total) → call InventoryService reserve stock (CompletableFuture) → if success publish OrderConfirmedEvent (Event Hubs) → if fail publish OrderFailedEvent + rollback inventory. Compensating transactions sur failure.

\subsection{Validations}
Order : total >= 0, items not empty. OrderItem : quantity > 0, unit\_price > 0. Address : no null required fields. @Validated annotation service level.

\subsection{Transactions}
@Transactional on service methods. Propagation REQUIRED. Isolation READ\_COMMITTED. Rollback on unchecked exceptions. Manual rollback optionnel.

\section{Inventory Service}
\subsection{Présentation générale}
Le Inventory Service assure la gestion des quantités physiques et réservées. Il expose des APIs pour vérifier la disponibilité, réserver du stock lors d'une commande et libérer le stock en cas d'échec du paiement ou d'annulation. Il conserve un historique des mouvements pour l'audit et la traçabilité.

\subsection{Fonctionnalités principales}
\begin{itemize}
	\item Consultation de l'état des stocks (on-hand, reserved, available).
	\item Réservation atomique de quantités pour assurer la cohérence pendant la création de commande.
	\item Mécanismes de réapprovisionnement et alertes de stock critique.
	\item Journalisation des mouvements de stock (audit trail) pour traçabilité.
\end{itemize}

\subsection{Endpoints}
GET /api/inventory/:product\_id (stock status). POST /api/inventory/reserve (product\_id, quantity, reference\_id). POST /api/inventory/release (reference\_id). GET /api/inventory/movements (audit trail). POST /api/inventory/restock (admin).

\subsection{Entités}
Inventory : product\_id (PK), quantity\_on\_hand, quantity\_reserved, quantity\_available (calculated). StockMovement : id, product\_id (FK), movement\_type (IN/OUT/RESERVED/RELEASED), quantity, reason, reference\_type (ORDER/MANUAL), reference\_id, created\_at.

\subsection{Event Consumption}
Consume OrderCreatedEvent → reserve stock. Consume PaymentFailedEvent → release reserved stock. Error handling dead-letter queue on processing failure.

\section{Notification Service}
\subsection{Présentation générale}
Le Notification Service centralise l'envoi de messages aux utilisateurs (emails, webhooks, push). Il consomme les événements métiers (OrderConfirmed, OrderShipped, StockCritical) et orchestre l'envoi en s'appuyant sur adaptateurs externes (SendGrid, SMTP, services push). Il gère les templates et les politiques de retry pour garantir la remise.

\subsection{Fonctionnalités principales}
\begin{itemize}
	\item Envoi d'emails transactionnels (confirmation commande, expédition, réinitialisation mot de passe).
	\item Support des templates paramétrés et internationalisation.
	\item Gestion des retries et des erreurs (backoff exponentiel, dead-letter pour notifications non remises).
	\item API utilisateur pour lister et marquer les notifications comme lues.
\end{itemize}

\subsection{Endpoints}
GET /api/notifications (user notifications). PUT /api/notifications/:id/read. DELETE /api/notifications/:id.

\subsection{Event Consumers}
Listen OrderConfirmedEvent → send order confirmation email (template). Listen OrderShippedEvent → send shipment notification email (with tracking#). Listen StockCriticalEvent → send admin alert.

\subsection{Email Integration}
SendGrid client (optionnel : Azure Mail). Template : order confirmation (order#, items, total), shipment (tracking URL), password reset. Retry logic exponential backoff 3 attempts. Failure logging sans blocking main process.

\section{Exception Handling}

\subsection{Custom Exceptions}
EntityNotFoundException : 404 response. ValidationException : 400 response. UnauthorizedException : 401 response. ForbiddenException : 403 response. BusinessException : 422 response.

\subsection{Global Exception Handler}
@ControllerAdvice GlobalExceptionHandler. Methods @ExceptionHandler per exception type. Response : errorCode, message, timestamp, path. Logging all exceptions ERROR level.

\section{Validation Framework}

Bean Validation annotations : @NotBlank, @Email, @Min(1), @Size(min, max). Custom validators : @UniqueEmail, @ValidAddress. Validation groups optionnel (create vs update).

\section{Data Access Patterns}


L’accès aux données dans le backend repose principalement sur le pattern Repository, qui permet d’abstraire la logique d’accès aux bases de données. Pour chaque entité métier, un repository dédié est défini, exposant des méthodes standards telles que \texttt{save}, \texttt{findById}, \texttt{findAll} ou \texttt{deleteById}. Cette approche favorise la réutilisabilité et la clarté du code, tout en permettant l’extension par des méthodes de requête personnalisées, nommées selon les attributs recherchés (ex : \texttt{findByEmail}, \texttt{findByStatus}). La gestion de la pagination et du tri est assurée par l’utilisation de l’objet \texttt{PageRequest}, qui permet de contrôler efficacement le volume de données retourné et leur ordre d’affichage.

Pour les besoins plus avancés, l’accès direct à l’Entity Manager est possible. Cela permet d’exécuter des requêtes complexes, d’utiliser des requêtes nommées (\texttt{@NamedQuery}) ou encore de recourir à des requêtes SQL natives lorsque les fonctionnalités du repository ne suffisent pas. Cette flexibilité garantit que même les scénarios d’accès aux données les plus exigeants peuvent être couverts sans compromettre la maintenabilité du code.

\section{Asynchronous Processing}

Le traitement asynchrone est largement utilisé pour améliorer la réactivité et la scalabilité des microservices. Les méthodes annotées avec \texttt{@Async} sont exécutées dans un thread séparé, ce qui permet de libérer le thread principal et d’optimiser la gestion des ressources. Le retour d’un \texttt{CompletableFuture} permet de chaîner des opérations et de gérer les résultats de façon non bloquante. Un pool de threads dédié est configuré pour contrôler le parallélisme et éviter la saturation du système. Des mécanismes de gestion des délais (timeout) et de rappel en cas d’erreur (error callback) sont également mis en place pour garantir la robustesse des traitements asynchrones.

\section{Caching Strategy}

La stratégie de cache vise à optimiser les performances et à réduire la charge sur les bases de données. L’annotation \texttt{@Cacheable} est utilisée pour mettre en cache les résultats des requêtes fréquentes, comme la récupération d’un produit par son identifiant, avec une durée de vie (TTL) de 10 minutes. Lorsqu’une mise à jour ou une suppression intervient, l’annotation \texttt{@CacheEvict} permet d’invalider le cache concerné afin de garantir la cohérence des données. L’annotation \texttt{@CachePut} est employée lors de l’insertion de nouveaux éléments pour les ajouter directement au cache. Pour les environnements distribués, l’intégration de Redis permet de partager le cache entre plusieurs instances de microservices, assurant ainsi une scalabilité horizontale et une haute disponibilité.

\section{Logging \& Monitoring}

La supervision et la traçabilité des applications sont assurées par une stratégie de logging et de monitoring avancée. Les logs sont produits au format JSON structuré grâce à SLF4J et Logback, ce qui facilite leur centralisation et leur analyse par des outils comme Loki. Chaque log contient des informations détaillées telles que le timestamp, le niveau de gravité, le service concerné, le pod d’exécution, les identifiants de trace et de span (pour la corrélation distribuée), l’identifiant utilisateur, le message et, en cas d’erreur, la stack trace complète. L’utilisation du Mapped Diagnostic Context (MDC) permet d’enrichir dynamiquement les logs avec des informations contextuelles. Côté monitoring, Micrometer expose des métriques détaillées sur les requêtes HTTP (nombre, latence), les requêtes base de données (nombre, durée) et l’efficacité du cache (hits/misses), permettant une observabilité fine et proactive du système.

\section{Build, Containerisation et Déploiement}
Le cycle de vie des microservices backend s’appuie sur une chaîne d’outils moderne et automatisée. La compilation, l’exécution des tests et le packaging sont réalisés avec Maven, qui garantit la reproductibilité des builds et la gestion efficace des dépendances. Chaque service dispose de son propre Dockerfile, conçu selon le principe du multi-stage build : la compilation s’effectue dans une image JDK légère, puis l’exécutable est transféré dans une image JRE minimale pour optimiser la taille et la sécurité des conteneurs.

Lors de l’intégration continue, le Maven Wrapper et le cache du repository local sont utilisés pour accélérer les builds et limiter la consommation de bande passante. Les images Docker générées sont systématiquement taguées avec le numéro de build Jenkins, assurant ainsi la traçabilité et la possibilité de rollback. Enfin, chaque service expose des endpoints Actuator (/actuator/metrics, /actuator/health) qui facilitent l’orchestration, le monitoring et l’intégration avec les outils de supervision du cluster Kubernetes.


\section{Base de données et Migrations}

Le backend utilise principalement MySQL (relationnel) et CosmosDB/MongoDB (NoSQL) pour la persistance des données. Les migrations sont gérées par Flyway (fichiers \texttt{src/main/resources/db/migration/}).

	extbf{Tables principales MySQL~:}
\begin{itemize}
	\item \textbf{t\_users}~: Stocke les informations d'authentification et de profil utilisateur (\texttt{id}, \texttt{username}, \texttt{email}, \texttt{password}, \texttt{role}, \texttt{enabled}). Cette table est centrale pour la gestion des utilisateurs et la sécurité. Elle est liée à \texttt{t\_orders} via la colonne \texttt{user\_id} (relation N:1, un utilisateur peut avoir plusieurs commandes).
	\item \textbf{t\_orders}~: Détaille les commandes passées (\texttt{id}, \texttt{order\_number}, \texttt{user\_id}, \texttt{sku\_code}, \texttt{price}, \texttt{quantity}). Chaque commande référence un utilisateur et un produit en stock.
	\item \textbf{t\_inventory}~: Gère le stock des produits (\texttt{id}, \texttt{sku\_code}, \texttt{quantity}).
	\item \textbf{t\_notifications}~: Historise les notifications envoyées (\texttt{id}, \texttt{order\_id}, \texttt{notification\_type}, \texttt{status}, \texttt{recipient}, \texttt{sent\_at}). Chaque notification est liée à une commande.
\end{itemize}

	extbf{Relations principales~:}
\begin{itemize}
	\item \texttt{t\_orders.user\_id} $\rightarrow$ \texttt{t\_users.id} (N:1)
	\item \texttt{t\_orders.sku\_code} $\rightarrow$ \texttt{t\_inventory.sku\_code} (N:1)
	\item \texttt{t\_notifications.order\_id} $\rightarrow$ \texttt{t\_orders.id} (N:1)
\end{itemize}

	extbf{Collection principale MongoDB~:}
\begin{itemize}
	\item \textbf{product}~: Catalogue produit (\texttt{_id}, \texttt{name}, \texttt{description}, \texttt{skuCode}, \texttt{price}, \texttt{category}, \texttt{images}, ...).
\end{itemize}

Recommandations~: sauvegardes régulières (point-in-time), surveillance du stockage, tests de restauration automatisés.


\section{API Contracts et Exemples}
Les endpoints exposent des contrats REST/JSON. Exemple simplifié de création de commande :
\begin{verbatim}
POST /api/orders
{
	"userId": 123,
	"items": [{"productId": "abc", "quantity": 2}],
	"paymentMethod": "card"
}
\end{verbatim}

Réponses : code 201 + payload contenant l'id de commande et le statut initial (PENDING).


\section{Pattern de Résilience et Communication Asynchrone}
Le projet implémente plusieurs patterns de résilience :
\begin{itemize}
	\item Circuit Breaker (Resilience4j) pour isoler dépendances défaillantes.
	\item Retry avec backoff exponentiel pour appels réseau transitoires.
	\item Timeout et fallbacks au niveau des clients HTTP.
\end{itemize}

Pour l'intégration asynchrone, \textbf{Event Hubs / Kafka} est utilisé pour publier des événements métiers (OrderCreated, OrderConfirmed, StockReserved). Les consommateurs gèrent la compensation (saga) en cas d'échec transactionnel.


\section{Tests - stratégie étendue}
\begin{itemize}
	\item \textbf{Unitaires (JUnit/Mockito)} : tests isolés pour services, validateurs et utilitaires.
	\item \textbf{Intégration (Spring Boot Test / Testcontainers)} : démarrage de slices applicatives et dépendances (MySQL, MongoDB) via Testcontainers.
	\item \textbf{Contract Tests (Spring Cloud Contract)} : stubs pour garantir compatibilité entre services (ex : api-gateway ↔ product-service).
	\item \textbf{End-to-end} : scénarios avec l'environnement Docker Compose utilisé par Jenkins pour valider flux critiques (checkout → payment → notification).
\end{itemize}

\section{Sécurité spécifique au backend}
- JWT validation via filtros Spring Security, vérification signatures RS256.
- Limitation rate-limiting au niveau de l'API Gateway pour protéger endpoints critiques.
- Vérification des inputs via Bean Validation et DTOs immuables pour prévenir injection.

\section{Instrumentation et Observabilité applicative}
Chaque service publie :
\begin{itemize}
	\item métriques Micrometer (http.server.requests, jdbc.connections.active),
	\item traces (si Tempo/Jaeger activé) via headers W3C TraceContext,
	\item logs structurés envoyés à Loki via \texttt{loki-logback-appender}.
\end{itemize}

\section{Performances et scalabilité}
Recommandations pratiques :
\begin{itemize}
	\item Taille initiale du pool Hikari calculée en fonction du nombre de connexions attendues et du nombre de réplicas pods.
	\item Autoscaling Horizontal (HPA) configuré sur métriques CPU et custom metrics (http queue length).
	\item Tests de charge (k6, Gatling) sur endpoints critiques, seuils d'alerte p95/p99 définis dans Grafana.
\end{itemize}






% Police uniforme : lmodern
\chapter{Application Frontend}

\section{Présentation}
L'interface utilisateur est développée en \textbf{Angular 18} et TypeScript. Elle couvre les parcours principaux : navigation catalogue, fiche produit, panier, paiement, profil utilisateur et tableau d'administration.

\section{Dashboards et gestion des rôles}
L'application frontend propose deux tableaux de bord distincts selon le type d'utilisateur connecté~:
\begin{itemize}
	\item \textbf{Dashboard Administrateur}~: Accessible uniquement aux utilisateurs ayant le rôle \texttt{admin}. Ce dashboard permet de gérer l'ensemble de l'application~: ajout/modification/suppression de produits, visualisation du stock en temps réel, gestion des commandes, accès à l'historique des notifications, et administration des utilisateurs. L'admin dispose d'une vue synthétique sur les métriques clés (ventes, stocks, utilisateurs actifs).
	\item \textbf{Dashboard Utilisateur}~: Pour les clients classiques. Ce dashboard permet de parcourir le catalogue, consulter les fiches produits, ajouter au panier, passer commande et suivre l'état de ses commandes. L'utilisateur peut également gérer son profil et consulter l'historique de ses achats et notifications.
\end{itemize}
La navigation et l'affichage des fonctionnalités sont dynamiquement adaptés selon le rôle après authentification.

\section{Organisation du projet}
- \texttt{src/app/} : modules et routes.
- \texttt{components/} : composants réutilisables (cart, product-card, navbar).
- \texttt{services/} : clients API (product, order, auth, notification).
- \texttt{guards/} : protection des routes et contrôle d'accès.
- \texttt{environments/} : variables d'environnement (URL API, flags).

\section{Authentification}
- Authentification basée sur JWT ; le frontend doit privilégier les cookies HttpOnly pour la sécurité.
- Intercepteur HTTP pour l'injection du token et la gestion centralisée des erreurs (401/403).
- Guards pour protéger les routes sensibles (admin).

\section{Ergonomie et accessibilité}
- Design responsive mobile/desktop.
- Bonnes pratiques d'accessibilité : aria‑labels, gestion du focus, contraste des couleurs.
- Améliorations UX : chargement optimiste, indicateurs de progression et notifications utilisateur.

\section{Tests}
- Unitaires : Karma + Jasmine.
- E2E : Cypress pour les scénarios critiques (login, achat, administration).
- Commandes utiles : \texttt{npm test}, \texttt{npm run build}, \texttt{npm run test:e2e}.






% Police uniforme : lmodern
\chapter{Monitoring et Observabilité}

\section{But}
Assurer visibilité, détection d'incidents et capacité à diagnostiquer rapidement (métriques, logs, traces).

\section{Composants déployés}
- \textbf{Prometheus} : collecte des métriques (scrape des endpoints actuator des services).
- \textbf{Grafana} : dashboards métiers et opérationnels, alerting.
- \textbf{Loki} : centralisation des logs structurés (JSON).
- \textbf{Tempo} (optionnel) : traces distribuées.

\section{Configuration observée}
Le dépôt inclut une configuration Prometheus qui scrape les endpoints exposés par les services (API Gateway, product, order, inventory, notification). Les intervalles et rétentions sont adaptés pour l'environnement (tests locaux vs production).

\section{Dashboards et alertes}
Exemples de dashboards : santé du cluster (CPU/mémoire), performance API (p50/p95/p99), KPI métier (commandes/heure). Règles d'alerte : pod down, latence p99 excessive, erreurs 5xx élevées, DB pool exhausted. Les alertes sont routées vers Slack, email, et PagerDuty.

\section{Logs et corrélation}
- Logs JSON standardisés : timestamp, niveau, service, pod, trace\_id.
- Corrélation métriques ↔ logs ↔ traces via trace_id/span_id pour faciliter les RCA.

\section{Tests et validation}
- Tests d'alerte : simulation d'augmentation d'erreurs, perte de service, latence DB.
- Vérification des notifications (Slack/email) et des runbooks.



% Police uniforme : lmodern
\chapter{Infrastructure Cloud}

\section{Résumé}
L'infrastructure est définie via Terraform dans \texttt{Devops/terraform} pour assurer infrastructure as code, traçabilité et réplicabilité. La cible principale est Microsoft Azure.

\section{Ressources principales (extrait Terraform)}

\subsection{Resource Group}
Le resource group \texttt{rg-ecom-dev} regroupe l'ensemble des ressources Azure du projet. Il permet une gestion centralisée, la traçabilité et la suppression atomique de l'environnement.

\subsection{Azure Container Registry (ACR)}
L'ACR est utilisé pour stocker et versionner les images Docker de tous les microservices (backend, frontend). Chaque build CI/CD pousse une image taguée (ex : \texttt{product-service:build123}) dans le registre. Cela garantit la reproductibilité des déploiements et la sécurité des artefacts.

\subsection{Azure Kubernetes Service (AKS)}
Le cluster AKS orchestre le déploiement, la montée en charge et la résilience des conteneurs applicatifs. Il est configuré avec 2 nœuds (VM Standard\_D2s\_v3) pour supporter la charge de développement et de test. AKS gère l'autoscaling, le réseau, la sécurité (RBAC), et l'intégration avec ACR pour le pull d'images.

\subsection{MySQL Flexible Server}
Le serveur \texttt{mysql-ecom} héberge la base de données transactionnelle du service de commandes (order-service). Il est provisionné avec 7 jours de backup, la base \texttt{ecomdb}, et une configuration adaptée à la haute disponibilité. Les accès sont sécurisés (login admin, mot de passe stocké dans Key Vault). Les règles firewall sont ouvertes pour le développement mais doivent être restreintes en production.

\subsection{Azure Key Vault}
Le Key Vault centralise la gestion des secrets, mots de passe, et clés d'API. Il protège les informations sensibles utilisées par les applications (ex : credentials DB, secrets JWT, clés d'API externes). L'accès est contrôlé par RBAC Azure et la soft delete est activée pour la récupération en cas de suppression accidentelle.

\subsection{Azure Storage Account}
Le compte de stockage \texttt{stgecomdev} héberge les fichiers statiques et les images produits uploadées via le frontend. Le conteneur \texttt{product-images} est utilisé par le Product Service pour stocker et servir les images associées aux produits. L'accès se fait via des URLs sécurisées et le stockage est répliqué pour la durabilité.

\subsection{Réseau et Sécurité}
Le cluster AKS utilise Azure CNI pour l'intégration réseau, avec un load balancer standard pour exposer les services. Le rôle \texttt{AcrPull} est attribué au kubelet pour autoriser le téléchargement des images depuis l'ACR. Les secrets sont injectés dans les pods via CSI driver recommandé. Le firewall MySQL est ouvert pour les tests mais doit être restreint à des IPs de confiance en production.

\section{Réseau et sécurité}
- AKS configuré avec Azure CNI et load balancer standard.
- Rôle \texttt{AcrPull} attribué au kubelet pour pull d'images ACR.
- Firewall MySQL ouvert pour tests (à restreindre en production).
- Secrets gérés via Key Vault, injection via CSI driver recommandée.

\section{Provisioning et workflow}
- Terraform modules séparés (main.tf, mysql.tf, keyvault\_storage.tf, etc.).
- Variables externalisées dans \texttt{terraform.tfvars} et pipeline CI réalise \texttt{terraform plan} puis \texttt{apply} après revue.

\section{Recommandations opérationnelles}
- Restreindre les règles firewall MySQL à plages IP connues.
- Activer rotation automatique et purge protection pour Key Vault en production.
- Compléter par Azure Monitor pour alerting natif.






% Police uniforme : lmodern
\chapter{LlmOps \& CI/CD}

\section{Pipeline CI/CD (Jenkins)}
Résumé des étapes observées :
\begin{enumerate}
  \item Checkout du dépôt.
  \item Backend : build Maven, exécution des tests unitaires et d'intégration.
  \item Frontend : tests Angular (Karma/Jasmine) et lint.
  \item Tests d'intégration end‑to‑end : docker-compose.test.yml pour exécuter la pile locale.
  \item Build d'images Docker pour chaque service et push vers ACR.
  \item Déploiement : authentification AKS puis déploiement via Helm (\texttt{helm upgrade --install --atomic}).
  \item Post‑déploiement : validations de santé et récupération des URLs.
\end{enumerate}

\section{Scripts d'automatisation}
Scripts pour installer CRDs Prometheus, récupérer URLs de services, initialiser AKS/AKS credentials et déployer Helm avec valeurs provenant de Key Vault.

\section{Sécurité dans la chaîne de livraison}
- Scans SCA/SAST : Snyk/Trivy/SonarQube intégrés.
- Scans d'images au push vers ACR.
- Gestion d'identités via comptes de service et RBAC Kubernetes.

\section{Observations}
Le pipeline est conçu pour être reproductible et sécurisé ; points à vérifier : gestion des secrets en CI, permissions minimales pour les service accounts, politique de rollback et tests de reprise.


\section{Détails du pipeline Jenkins}
Le pipeline principal est défini dans \texttt{Devops/jenkins/Jenkinsfile}. Il est structuré en plusieurs étapes clés, chacune jouant un rôle essentiel dans la chaîne CI/CD. Voici le détail de chaque phase :

\subsection{1. Initialisation et préparation de l'environnement}
Cette étape prépare l'environnement d'exécution du pipeline :
\begin{itemize}
  \item Chargement des variables d'environnement depuis \texttt{jenkins.env}.
  \item Authentification auprès d'Azure via un Service Principal sécurisé.
  \item Récupération des identifiants du registre de conteneurs Azure (ACR).
  \item Configuration de \texttt{kubectl} pour interagir avec le cluster AKS.
  \item Installation de Helm pour la gestion des déploiements Kubernetes.
\end{itemize}
Cette phase garantit que toutes les dépendances et accès nécessaires sont en place avant de lancer les builds et déploiements.

\subsection{2. Tests unitaires et d'intégration}
Le pipeline sépare les tests en plusieurs sous-étapes :
\begin{itemize}
  \item \textbf{Backend} : Compilation Maven, exécution des tests unitaires (JUnit/Mockito) et des tests d'intégration (Spring Boot Test, Testcontainers).
  \item \textbf{Frontend} : Lancement des tests Angular avec Karma/Jasmine, ainsi que l'analyse statique (lint).
  \item \textbf{Tests end-to-end} : Utilisation de \texttt{docker-compose.test.yml} pour démarrer l'ensemble des services dans un environnement isolé et exécuter des scénarios de bout en bout.
\end{itemize}
Chaque étape de test produit des rapports qui sont archivés comme artifacts Jenkins pour analyse ultérieure.

\subsection{3. Construction des images Docker}
Pour chaque microservice (backend et frontend), une image Docker est construite :
\begin{itemize}
  \item Utilisation de Docker multi-stage pour optimiser la taille des images.
  \item Les images sont taguées avec le numéro de build Jenkins (\texttt{IMAGE\_TAG}) pour assurer la traçabilité.
  \item Exemple de commande : \texttt{docker build -f product-service/Dockerfile -t $ACR\_NAME/product-service:$IMAGE\_TAG .}
\end{itemize}
Cette étape garantit que chaque service dispose d'une image prête à être déployée et versionnée.

\subsection{4. Push des images vers Azure Container Registry (ACR)}
Les images Docker construites sont poussées vers le registre ACR :
\begin{itemize}
  \item Authentification sécurisée auprès d'ACR.
  \item Push de chaque image avec son tag unique.
  \item Vérification de la présence des images dans le registre avant déploiement.
\end{itemize}
Cette phase permet de centraliser les artefacts de déploiement et de faciliter la gestion des versions.

\subsection{5. Déploiement sur AKS avec Helm}
Le déploiement s'effectue via Helm, orchestrateur de packages Kubernetes :
\begin{itemize}
  \item Exécution de \texttt{helm upgrade --install --atomic} pour déployer ou mettre à jour les services.
  \item Injection dynamique des variables sensibles (secrets, URLs) depuis Azure Key Vault ou les variables Jenkins.
  \item Activation des probes de santé (liveness/readiness) pour valider le bon fonctionnement des pods.
  \item Utilisation de l'option \texttt{--atomic} pour garantir le rollback automatique en cas d'échec du déploiement.
\end{itemize}
Cette étape assure un déploiement fiable, sécurisé et automatisé sur le cluster Kubernetes.

\subsection{6. Post-déploiement et vérifications}
Après le déploiement, le pipeline effectue plusieurs contrôles :
\begin{itemize}
  \item Vérification de l'état des pods et des endpoints exposés.
  \item Exécution de smoke tests (vérification des endpoints santé, création de commande de test, etc.).
  \item Génération d'un fichier \texttt{service-urls.txt} listant les URLs des services déployés.
  \item Archivage des rapports de tests et des artefacts de déploiement dans Jenkins.
\end{itemize}
Ces vérifications permettent de s'assurer que l'application est opérationnelle et que le déploiement s'est déroulé sans incident.

\subsection{7. Bonnes pratiques et sécurité}
\begin{itemize}
  \item Utilisation d'agents Docker dédiés pour isoler les environnements de build (image Maven, image Node).
  \item Gestion des identifiants sensibles via le store de credentials Jenkins (jamais commités dans le repo).
  \item Génération dynamique du fichier \texttt{jenkins.env} par les scripts d'installation, référencé via \texttt{configFileProvider}.
  \item Utilisation du cache Maven (\texttt{~/.m2}) pour accélérer les builds.
  \item Définition de politiques de rollback automatiques et de probes de santé pour fiabiliser les déploiements.
\end{itemize}
L'ensemble de ces pratiques contribue à la robustesse, la sécurité et la maintenabilité de la chaîne CI/CD.

\section{Terraform et gestion d'état}
Le code Terraform est situé dans \texttt{Devops/terraform}. Observations et recommandations :
\begin{itemize}
  \item Le plan crée un Resource Group, ACR et AKS avec identité managée. Les fichiers clés : \texttt{main.tf}, \texttt{mysql.tf}, \texttt{keyvault_storage.tf}, \texttt{eventhubs.tf}.
  \item Utiliser un backend distant pour l'état (Azure Storage + blob lock) au lieu de laisser \texttt{terraform.tfstate} localement; cela évite les conflits d'équipes.
  \item Intégrer \texttt{terraform fmt} et \texttt{terraform validate} dans la pipeline et exiger une revue des plans (gitops ou approbation manuelle) avant \texttt{apply} en production.
\end{itemize}

\section{Sécurité de la chaîne de livraison}
\begin{itemize}
  \item Scans SCA (Snyk) et image scanning (Trivy) automatisés lors de la construction des images. Les builds qui échouent sur vulnérabilités critiques doivent être bloqués.
  \item Rotation des secrets et intégration Key Vault via scripts (\texttt{Devops/scripts/populate-keyvault.*}).
  \item Principe du moindre privilège pour les service principals et comptes Jenkins (AcrPull, Reader minimal, role assignments limités).
\end{itemize}

\section{Observabilité et vérifications post-déploiement}
Après déploiement :
\begin{itemize}
  \item Vérifier l'état des pods (liveness/readiness), endpoints exposés et logs initiaux.
  \item Valider la configuration Prometheus (ServiceMonitors) et l'adoption des CRDs avant déploiement du monitoring stack.
  \item Exécuter smoke tests (endpoints santé, requête GET catalogue, création commande de test).
\end{itemize}



\input{07_securite}
% --- Conclusion CI/CD adaptée au projet ---
        	ableofcontents
        \newpage
La mise en place d'une chaîne CI/CD complète avec Jenkins constitue un pilier fondamental de cette plateforme e-commerce. Le pipeline orchestre l'automatisation des tests (unitaires, d'intégration, de sécurité) pour le frontend (Angular 18) et le backend (microservices Spring Boot 3, Java 21), garantissant la qualité à chaque étape. La containerisation via Docker assure la cohérence des environnements du développement à la production. L'infrastructure as code, gérée par Terraform, permet le provisionnement automatisé et reproductible de toutes les ressources sur Azure (AKS, ACR, Key Vault, Storage, MySQL Flexible Server, etc.). Les déploiements Kubernetes sont industrialisés grâce à Helm, assurant la scalabilité, la haute disponibilité et la résilience des microservices. Le pipeline intègre également la configuration automatique d'une stack de supervision avancée (Prometheus, Grafana, Loki, Tempo, AlertManager) pour une observabilité complète et des alertes proactives. La sécurité est renforcée par la gestion centralisée des secrets (Azure Key Vault), l'automatisation des certificats SSL, l'analyse continue du code (SonarQube, Trivy) et la gestion des accès. Cette chaîne CI/CD accélère les cycles de livraison tout en assurant la robustesse, la sécurité et la fiabilité du système.

\section{Innovation Technique}

Ce projet se distingue par l'intégration avancée des pratiques DevOps et Cloud Native. L'architecture microservices découplée (Spring Boot, Angular) favorise l'évolutivité, la maintenabilité et le déploiement indépendant des services (product, order, inventory, notification). L'automatisation de l'infrastructure (Terraform, Helm) et des déploiements (Jenkins) permet une gestion agile et reproductible de l'ensemble du cycle de vie applicatif. L'observabilité est assurée par une stack de monitoring moderne (Prometheus, Grafana, Loki, Tempo) couvrant métriques, logs et traces distribuées. La sécurité est traitée de bout en bout : gestion des secrets, audits, scans de vulnérabilités, authentification JWT, et conformité RGPD. L'utilisation d'Azure AKS pour l'orchestration, associée à l'autoscaling, la tolérance aux pannes et le stockage distribué, garantit la résilience et la disponibilité de la plateforme. Enfin, la démarche IaC et CI/CD permet d'intégrer rapidement de nouvelles fonctionnalités tout en maintenant un haut niveau de qualité.

\section{Perspectives et Évolutions Futures}

Plusieurs axes d'amélioration sont identifiés pour renforcer la plateforme :
\begin{itemize}
    \item \textbf{Automatisation avancée} : Intégration de tests de charge automatisés, déploiement blue/green et canary pour des mises à jour sans interruption.
    \item \textbf{Sécurité renforcée} : Mise en place d'une architecture Zero Trust, détection d'anomalies par IA, gestion centralisée des identités et conformité accrue.
    \item \textbf{Scalabilité et performance} : Optimisation de l'autoscaling AKS, multi-région Azure pour la haute disponibilité mondiale, et cache distribué (Redis).
    \item \textbf{Expérience utilisateur} : Ajout de fonctionnalités temps réel (WebSocket), personnalisation avancée, et intégration de modules IA pour la recommandation de produits.
    \item \textbf{Observabilité proactive} : Détection prédictive des incidents, dashboards dynamiques, et alertes intelligentes.
    \item \textbf{Écosystème ouvert} : Développement d'API publiques, intégration de partenaires, et ouverture à des extensions tierces.
    \item \textbf{Gouvernance et conformité} : Traçabilité complète (audit), gestion fine des accès, automatisation des contrôles réglementaires et reporting avancé.
\end{itemize}
La base technique solide posée par ce projet permet d'envisager sereinement ces évolutions, avec une capacité d'innovation continue grâce à l'automatisation et à l'agilité DevOps.

\end{document}