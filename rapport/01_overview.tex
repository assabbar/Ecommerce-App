


% Police uniforme : lmodern
\chapter{Présentation du projet}

\section{Introduction}
Le projet présente une plateforme e-commerce basée sur une architecture microservices moderne destinée à fournir une expérience d'achat scalable, résiliente et maintenable. Le frontend est développé en \textbf{Angular 18} tandis que le backend est composé de services Java utilisant \textbf{Spring Boot 3} (Java 21). L'ensemble est containerisé avec \textbf{Docker}, orchestré avec \textbf{Kubernetes} et automatisé via \textbf{Terraform} et des pipelines CI/CD. \textbf{L'application est hébergée sur Microsoft Azure, déployée dans un cluster Kubernetes (AKS) pour garantir la haute disponibilité, la sécurité et la scalabilité.}

\section{Objectifs}
- Fournir une architecture modulaire permettant l'évolution indépendante des services.
- Assurer la haute disponibilité et la résilience via l'orchestration Kubernetes.
- Mettre en place une chaîne CI/CD reproductible et sécurisée pour accélérer les livraisons.
- Garantir l'observabilité complète (métriques, logs, traces) et la gestion centralisée des secrets.

\section{Vue d'ensemble de l'architecture}
Le flux de données général est le suivant :
\begin{enumerate}
  \item L'utilisateur navigue depuis l'interface Angular.
  \item Les requêtes sont routées via l'API Gateway qui gère l'authentification et le routage vers les microservices.
  \item Les microservices (product, order, inventory, notification) traitent la logique métier et communiquent éventuellement via des messages asynchrones.
  \item La supervision collecte métriques et logs (Prometheus, Grafana, Loki) et déclenche des alertes via AlertManager.
  \item L'infrastructure est provisionnée et versionnée via Terraform.
\end{enumerate}

\section{Cas d'utilisation}
- Utilisateur : navigation catalogue, ajout au panier, validation de commande, suivi.
- Administrateur : gestion des produits et stocks, surveillance des indicateurs et gestion des incidents.

\section{Technologies principales}

\subsection{Frontend}
\begin{table}[H]
\centering
\begin{tabular}{|l|p{9cm}|}
\hline
	extbf{Technologie} & \textbf{Rôle dans le projet} \\
\hline
Angular 18 & Framework principal pour le développement de l'interface utilisateur SPA, gestion du routage, des composants et de l'état. \\
TypeScript & Langage typé pour la robustesse, la maintenabilité et la sécurité du code frontend. \\
Tailwind CSS & Framework utilitaire pour le design responsive et la personnalisation rapide de l'UI. \\
Karma/Jasmine & Outils de tests unitaires pour garantir la qualité du code frontend. \\
Cypress & Tests end-to-end automatisés pour valider les parcours critiques utilisateur. \\
\hline
\end{tabular}
\end{table}

\subsection{Backend}
\begin{table}[H]
\centering
\begin{tabular}{|l|p{9cm}|}
\hline
	extbf{Technologie} & \textbf{Rôle dans le projet} \\
\hline
Java 21 & Langage principal pour la logique métier, la performance et la sécurité. \\
Spring Boot 3 & Framework pour la création de microservices REST, gestion de la configuration, des dépendances et de la sécurité. \\
Spring Data JPA & Accès et mapping aux bases de données relationnelles (MySQL). \\
Spring Data MongoDB & Accès aux données NoSQL pour certains services. \\
Spring Security & Authentification et autorisation (JWT, rôles). \\
Maven & Build, gestion des dépendances, packaging et exécution des tests. \\
JUnit/Mockito & Tests unitaires et d'intégration backend. \\
Testcontainers & Tests d'intégration avec bases de données et services réels en conteneur. \\
\hline
\end{tabular}
\end{table}

\subsection{Infrastructure}
\begin{table}[H]
\centering
\begin{tabular}{|l|p{9cm}|}
\hline
	extbf{Technologie} & \textbf{Rôle dans le projet} \\
\hline
Docker & Containerisation des applications pour portabilité et cohérence des environnements. \\
Kubernetes (AKS) & Orchestration, déploiement, scalabilité et gestion des microservices sur Azure. \\
Helm & Gestion des packages Kubernetes, déploiement automatisé et versionné. \\
Terraform & Infrastructure as Code pour provisionner et gérer toutes les ressources Azure. \\
Azure Container Registry (ACR) & Stockage et versionnement des images Docker. \\
Azure Key Vault & Gestion centralisée et sécurisée des secrets et credentials. \\
Azure Storage & Stockage des fichiers statiques et images produits. \\
MySQL Flexible Server & Base de données transactionnelle pour les commandes et l'inventaire. \\
\hline
\end{tabular}
\end{table}

\subsection{Monitoring et Observabilité}
\begin{table}[H]
\centering
\begin{tabular}{|l|p{9cm}|}
\hline
	extbf{Technologie} & \textbf{Rôle dans le projet} \\
\hline
Prometheus & Collecte des métriques applicatives et système (scraping des endpoints Actuator). \\
Grafana & Visualisation des métriques, création de dashboards et alerting. \\
Loki & Centralisation et recherche des logs applicatifs structurés. \\
Tempo & Tracing distribué pour l’analyse des performances et la corrélation des requêtes. \\
\hline
\end{tabular}
\end{table}

\subsection{CI/CD}
\begin{table}[H]
\centering
\begin{tabular}{|l|p{9cm}|}
\hline
	extbf{Technologie} & \textbf{Rôle dans le projet} \\
\hline
Jenkins & Orchestration des pipelines CI/CD, automatisation des builds, tests et déploiements. \\
Docker Registry (ACR) & Stockage sécurisé des images Docker pour déploiement sur AKS. \\
Helm charts & Déploiement automatisé et versionné des applications sur Kubernetes. \\
SonarQube & Analyse de la qualité et de la sécurité du code. \\
Trivy/Snyk & Scans de vulnérabilités sur les images et dépendances. \\
\hline
\end{tabular}
\end{table}

